\chapter{Synthesis of Bulk Metallic Glasses}

\section{Introduction}
The first amorphous metal was prepared by Duwez and his coworkers in 1960 with Au-Si alloy system by using splat quenching technique. In this experiment, the samples were rapidly quenched by achieving extremely highly cooling rates ~ 10$^6$ K/s. Later on, researchers discovered new classes of amorphous metals with more complex systems and multi-constituent elements. These new alloys were possible to fabricate with lower cooling rates ($\geq$ 100K/s). After recognizing the potential of amorphous metals in various applications and specially the use of soft magnetic properties in amorphous metal power distribution transformers, a tremendous research was carried out to improve fabrication techniques. In this regard, “melt spinning” was developed which resulted to produce high quality, thin and continuous ribbons of uniform thickness with isotropic properties. Various methods of rapid quenching were developed to produce ribbons as well as bulk samples in which single roller melt spinning, copper mold casting, suction casting and tilt casting are very famous for the production of metallic glasses. In this chapter we describe a few fabrication techniques which are utilized in the present times.

\section{Arc Melting and Induction Furnace} 
Preparations of master ingots for the production of BMGs need special care during the melting of constituent elements as the casting of glassy metals is very sensitive to contamination of impurities and oxidation. Inert gas plasma heating method (arc-melting) and induction heating are the two basic techniques used for the preparation of master ingots in the field of BMGs.
Inert gas plasma heating method or arc-melting method uses a continuous arc discharge to provide current flow between tungsten electrode and a refrigerated copper mold as an electrical ground. A flow of cooling water is always maintained to lower the temperature of copper hearth. Current intensities ranges from 100-150A to raise the temperature of alloy to 1500-2500K. A high inert atmosphere and getter process in the presence of Zr/Ti pellets avoids the oxidation of the master ingots. 
\\Induction heating method requires a radio frequency generator to power an inductance coil made of copper pipe. Cooling water always flow inside the copper pipe to lower its temperature. Boron nitride crucible is usually used as a vessel to keep alloying elements inside the induction coil. For the efficient heating, proper matching of the generator’s LC circuit impedance and coil inductance is tuned witch induces RF-currents around the crucible. The magnetic permeability of material to the RF-currents responses to eddy current heating. The whole process is carried out in highly pure (99.9999\%) inert gas atmosphere. Intensive care is devoted to keep the designed compositions of the samples. Master ingots are melted several times (at least 5 time each side) to get the homogeneous composition inside.

\begin{figure}[h]
\includegraphics[width=\linewidth]{arc2}
\caption{An optical image of the ingots on Cu-hearth during the fabrication process inside the twin arc-melter}
\end{figure}

\begin{figure}[h]
\begin{subfigure}{0.5\textwidth}
\includegraphics[width=0.9\linewidth, height=5cm]{arc1} 
\end{subfigure}
\begin{subfigure}{0.5\textwidth}
\includegraphics[width=0.9\linewidth, height=5cm]{induction}
\end{subfigure}
\caption{arc melting crucible(left) \& diagram of a induction heater}
\end{figure}

\section{Melt spinning} 
Melt spinning is more versatile and commonly used technique to produce thin amorphous ribbons. A melt spinner consists of a copper wheel fixed with an induction motor which could rotate it with extremely high speeds (~ 10,000rpm). The surface of the copper wheel is just a shiny flat plane. This massive copper wheel works as a heat sink for melt. The smoothness and cleanliness of wheel surface is important otherwise in contact surface of ribbon reflects the imperfections of the wheel. There is an induction heating coil made of copper pipe, with flowing cooling water to maintain the lower temperatures, situated just above the copper wheel. This induction coil with radio frequency generator could be used to melt the alloy samples. A quartz tube, as a crucible, could be fixed with a holder which translates along x, y and z-axis to fix it at proper location. One end of the quartz tube could be tuned to very small orifice with diameter 0.1-0.5mm. Initially the tube stays inside the coil to melt the ingots and then could be brought down to the vicinity of the wheel surface (0.1-0.5mm). A precise distance between the orifice of the tube and wheel surface is maintained to constrain the melt and to generate the stable melt flow by ejecting with low pressure of inert gas (argon). By controlling the chamber pressure, temperature of the melt, ejection pressure, diameter of the tube orifice and speed of wheel, ribbons of 20-50um thickness with few mm width and several meter length could be prepared with homogeneous dimensions.

\begin{figure}[h!]
\centering
\includegraphics[width=0.7\linewidth]{meltspin1}
\caption{A schematic diagram of melt spinner}
\end{figure}

\begin{figure}[h!]
\centering
\includegraphics[width=0.7\linewidth]{meltspin2}
\caption{An optical image of the melt-spinner used for the production of ribbon samples}
\end{figure}
\begin{figure}[h!]
\centering
\includegraphics[width=0.7\linewidth]{meltspin3}
\caption{Metallic Glass ribbon formed by Melt Spinning process}
\end{figure}


\section{Cu-Mold casting} 
After the invention of complex multi component metallic glass systems which could be fabricated in monolithic amorphous phase even with very low critical cooling rates, it is now possible to fabricate BMGs with larger three dimensional shapes by pouring the melt into Cu-molds. There are two basic techniques by which melt could be inserted into molds. 
\begin{itemize}
  \item inert gas pressure injection
  \item piston injection
\end{itemize} 
Inert gas (argon) is usually used to inject the melt into molds while in later case solid piston shoots the melt into copper molds. In both cases, induction melting and inert gas plasma arc heating could be utilized for melting the ingots while quartz tubes with tuned nozzle orifice or boron nitride crucible could be utilized for keeping the master ingots. 
\\Various models of Cu-mold casting have been designed by considering different conditions for different alloy systems. Anyhow, successful fabrication could be achieved by a particular casting technique. Tilt casting method could be used to prepare the samples with larger dimensions. In this method, one could melt the constituent elements with an inert gas plasma arc melting on a copper hearth and then pour the melt in-situ into Cu- mold by tilting the copper hearth. One of the advantages of this technique is that contamination of the master ingot could be reduced during the transferring process between the chambers. Also, the pouring position of melt in this technique is advantageous which reduces the chances of nucleation sites during the fabrication process. However in this melting method one should consider the precipitation of heterogeneous nucleation by incomplete melting at the bottom side contacted with copper hearth. 

\begin{figure}[h]
\centering
\includegraphics[width=\linewidth]{cu1}
\caption{Schematic diagram of the injection pressure Cu-mold casting technique. Optical image of 1cm bulk metallic glass prepared by tilt casting method is also presented}
\end{figure}

\section{Physical Vapour Deposition}

Physical vapor deposition (PVD) describes a variety of vacuum deposition methods used to deposit thin films by the condensation of a vaporized form of the desired film material onto various workpiece surfaces. Generally it takes three steps for the deposition process and these are 
\begin{enumerate}
  \item production of atomic/molecular entities.
  \item their transport environment.
  \item condensation on substrate
\end{enumerate} 
Deposition of the films starts with nucleation and growth process by making a network structure by giving rise to continuous films.
Here we describe specific vapor phase deposition technique.

\subsection{Pulse Laser Deposition}
Since the invention of Laser by 1965, it has been used to produce high temperature and dense plasma by evaporating the small amount of materials with high powered short laser pulses. In 1987, the discovery of high Tc superconducting thin film by Venkatesan and his co-workers by using PLD with low oxygen pressure environment and low processing temperature opened the new horizons for the materials scientists to utilize the advantages of this technique by depositing even with complex compounds.\\
Pulse laser deposition technique has numerous advantages for thin film investigations. The stoichiometry of the target composition in the deposited films is quite easy to replicate and also reactive depositions are very advantageous by using this technique. In one way, it is very flexible to deposit multilayer and other way; the simple operation procedure makes it versatile technique. Targets used for the depositions are relatively small and inexpensive and studies with compositional variation could be performed readily. The variety of films deposited by PLD ranges from high Tc superconductors to piezoelectric, ferroelectric films, semiconducting oxides and metallic films to study their tremendous properties. PLD technique itself has been paid great attention for understanding the deposition principles, the influence of various parameters such as laser density, laser pulse frequency, gas pressure etc. PLD is a highly versatile technique in the field of materials science.\\
Laser wavelength, pulse frequency, pulse energy, spot size of laser, and mode structure are the important parameters belongs to laser beam which may effects the growth of thin films during deposition. Also the substrate temperature, distance between target and substrate, pressure inside the chamber and deposition geometry such as on or off axis deposition also effects the quality of films.

\begin{figure}[h!]
\centering
\includegraphics[width=0.7\linewidth]{pld1}
\caption{One possible configuration of a PLD deposition chamber.}
\end{figure}

\begin{figure}[h!]
\centering
\includegraphics[width=0.7\linewidth]{pld2}
\caption{A plume ejected from a SrRuO3 target during pulsed laser deposition.}
\end{figure}
