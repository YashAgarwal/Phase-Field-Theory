\chapter{Conclusions}

\begin{enumerate}
\item We have developed a phase field model of a polycrystalline binary alloy by combining the Cahn–Hilliard model for a compositionally inhomogeneous alloy with a model of polycrystals (due to Fan and Chen).
\item We have used this model for studying GB effects on SD in systems in which the atomic mobility at GB is the same as that in the grain interior.
\item In systems in which the GB-energy of the a phase is lower than that of the $\mathbold{\beta}$ phase, the primary effect of a GB is the formation of alternating bands of the product phases parallel to the boundary. At the end of the decomposition, the microstructure exhibits these GB-bands coexisting with a normal SD microstructure in the grain interior.
\item The number of bands has been rationalized in terms of the driving force for band formation ($\mathbold{\gamma_\beta-\gamma_\alpha}$), and the strength of initial composition fluctuations. Formation of bands also effectively arrests grain growth.
\end{enumerate}