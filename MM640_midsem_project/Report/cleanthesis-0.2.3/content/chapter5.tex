\chapter{Magnetic Properties and Applications}

\section{Magnetic Properties}
In many applications it is undesirable for the core to retain magnetization when the applied field is removed. This property, called hysteresis can cause energy losses in applications such as transformers. Therefore 'soft' magnetic materials with low hysteresis loss, such as silicon steel, rather than the 'hard' magnetic materials used for permanent magnets, are usually used in cores.
A magnetic core is a piece of magnetic material with a high permeability used to confine and guide magnetic fields in electrical, electromechanical and magnetic devices such as electromagnets, transformers, electric motors, generators, inductors, magnetic recording heads, and magnetic assemblies. Its high permeability, relative to the surrounding air, causes the magnetic field lines to be concentrated in the core material. The magnetic field is often created by a coil of wire around the core that carries a current. The presence of the core can increase the magnetic field of a coil by a factor of several thousand over what it would be without the core.
\\
Glassy metals are extremely soft magnets used as a core material in power distribution transformers while they also exhibit hard magnetism. The outstanding magnetic softness with high resistivity as compared to crystalline counter parts and other favourable characteristics such as frequency response of nearly zero magnetostriction\footnote{ Magnetostriction is a property of ferromagnetic materials that causes them to change their shape or dimensions during the process of magnetization. The variation of materials magnetization due to the applied magnetic field changes the strain due to the magnetic force until reaching its saturation value} alloys makes the amorphous materials ideal for magnetic cores and electronic devices. Highly magnetostrictive alloys can be used as excellent sensing elements for detecting stresses. Amorphous alloys are used as magnetic field detectors, in encoding devices, and in antitheft systems.
Amorphous alloys consist of a random array of atoms, and are normally considered to be isotropic on a long range scale. As a consequence, the ferromagnetic magnetization vector does not contain any type of special anisotropy. It provides a strong drive for the development of soft magnetic materials, and in fact excellent soft magnetic properties were achieved in various amorphous ferromagnetic materials based on iron, cobalt and nickel alloys.
\section{Magnetic Applications}
\begin{enumerate}
\item Metallic glasses are used as transformer core material in high power transformers.
\item Due to its high electrical resistivity and nearly zero temperature coefficient of resistance, these materials are used in making.
\item As the magnetic property of metallic glass are not affected by irradiation they are used in making container for nuclear waste disposal.
\item These materials are used in preparation of magnets for fusion reactors and magnets for leviated trains etc.
\item	Metallic glasses can also be used for making watch cases to replace Ni and other metals which can cause allergic reactions.
\end{enumerate}

\begin{figure}[h!]
\centering
\includegraphics[width=\linewidth]{m1}
\caption{They are used in core  of high power transformers.
}
\end{figure}

\begin{figure}[h!]
\centering
\includegraphics[width=\linewidth]{m2}
\caption{They are used in tape recorder heads.
}
\end{figure}

\begin{figure}[h!]
\centering
\includegraphics[width=  \linewidth]{m3}
\caption{Magnetic Shielding: Fe-based BMGs exhibit a constant $\mu$ of $\approx$110 for frequencies up to 10 MHz, comparable to commercial cores}
\end{figure}

