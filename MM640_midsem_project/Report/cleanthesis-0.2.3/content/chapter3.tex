

\chapter{Glass Forming ability(GFA) of alloys}

\section{introduction}
The glass forming ability (GFA) of a material is the potential of an alloy melt to form the glassy phase by passing through or suppressing crystalline phase during the solidification process. It is a competing process between liquid phase and the crystalline phases. One of the fundamental and long-standing problems of glassy metals is the low GFA. For the development of sustainable products, the understanding of glass formation along with GFA is the key to developing new BMGs with superior properties. Glass forming ability could be measured by examining the critical thickness (D$_{max}$) of the fabricated glass or by measuring the critical cooling rates (R$_c$) required to produce them in bulk form. An alloy possessing higher value of GFA could be achieved in larger dimensions of BMGs and vice versa. Measuring cooling rates of alloy melts during solidification is difficult, while D$_{max}$ of BMGs depends on method used for the fabrication. However, a lot of efforts have been devoted to develop different parameters to assess the GFA of the alloys. To understand why some compositions lasts in glassy phase and other are hardly possible, there have been many criteria’s proposed to predict the GFA by indicating about the formation of glass but kinetics of glass formation was not considered. 
\\However, recently, by considering the kinetics processes (the crystal growth rate, the nucleation rate, or transformation kinetics) some simple glass forming parameters were derived which significantly improve the progress in this area of research. In this chapter we define some of the very basic and important parameters which could help to predict the GFA of alloys. All of these parameters use transition temperatures in one or other way by considering kinetic processes, crystal growth rates, transformation kinetics. 

\section{Reduced glass transition temperature: $
T_{rg}$ }
$$
T_{rg} = \frac{T_g}{T_l}
$$ 
An important and earliest parameter for the prediction of GFA of alloys is the reduced glass transition temperature (T$_{rg}$) defined as T$_g$/T$_l$ where T$_g$ \& T$_l$ are the glass transition temperature and liquidus temperature of the alloy respectively. When the liquid alloy is cooled from the molten state down to T$_g$, the viscosity of the melt increases to a high value (1012 Pa-s) and as a consequence glass is formed. Therefore, higher values of T$_g$ and lower values of T$_l$ (or higher values of T$_{rg}$) for alloy composition are always favorable for the formation of glassy phase. 
\\
It is concluded that if the value of T$_{rg}$ is greater than 2/3 then homogeneous crystal nucleation will be essentially suppressed and formation of crystalline phases are hardly possible. Typically, the T$_{rg}$ value of the known metallic glasses varies in the range of 0.4-0.7. This criteria only considers the condition under which glass is formed without the accounting the thermal stability of BMGs. 
\section{Supercooled region ($\Delta T_x$)}
$$ \Delta T_x= T_x-T_g$$
The supercooled region ($\Delta T_x$) is the difference in the onset of crystallization temperature (T$_x$) and glass transition temperature (T$_g$). It was suggested that BMGs with wider $\Delta T_x$ are more resistant to crystallization by exhibiting adequate thermally stability. On this basis, it was proposed that GFA of the BMGs scales up with the $\Delta T_x$. By various studies it came to know that T$_g$ and T$_l$ are independent of thickness and the production method of glassy metals by revealing that it only considers the thermal stability without considering the ease of glass formation.\\
